\documentclass{article}
\usepackage{amsmath}
\title{Problemi di Meccanica Noti}
\author{Tommaso Pedroni}
\usepackage[italian]{babel}
\begin{document}
\maketitle
\tableofcontents
\clearpage
\section{Statica}
\subsection{Fune appesa 2.1}
La fune è appesa dall'alto e penzola verso il basso sotto
l'effetto della forza di gravità. La fune ha lunghezza $L$ e densità lineare $\rho$. Per trovare la tensione, basta 
prendere un frammento di corda $dl$, tale frammento è sotto l'effetto di tre forze: 
la forza peso $dl\rho g$, la tensione nel punto $l$ $T(l)$ e la tensione nella punto finale $T(l+dl)$.
Ogni frammento è in equilibrio.
\begin{equation}
    \begin{aligned}
        dl\rho g + T(l+dl) &= T(l) \\
        T(l+dl) - T(l) &= -dl\rho g \\
        T'(l) &= -\rho g \\
        T(l) &= -l\rho g \\
    \end{aligned}
\end{equation}

\subsection{Fune attorno a un polo p.26}
Una fune è tesa lungo a uno spicchio di circonferenza di angolo $\theta$, si applica una tensione $T_0$ a un estremo.
L'altra parte è attaccata a un oggetto pesante inamovibile, come una barca. Pongo il coefficiente di attrito statico $\mu$, calcolare 
la massima forza che si esercita sulla barca. \\
Poniamo un angolo $d\theta$, la fune su questo arco ha una tensione $T$ e la reazione vincolare di $N_{d\theta}$. Questa
forza bilancia le componenti interne della tensione $T \sin{d\theta /2}$.
Per angoli piccoli possiamo dire che $N_{d\theta} = Td\theta$. La forza di attrito agisce tangenzialmente alla fune grazie a questa componente normale.

\begin{equation}
    \begin{aligned}
        T(\theta + d\theta) &\le T(\theta) + \mu T d\theta \\
        dT &\le \mu T d \theta \\
        \ln{T} &\le \mu \theta + C \\
        T &\le T_0 e^{\mu \theta}
    \end{aligned}
\end{equation}

\subsection{Catenaria 2.8}
Considero il tratto $ds$ come un punto materiale. Chiamo il punto della catenaria in $s$ $P(s)$, dove s è l'arco della catenaria.
La forza peso in ogni tratto è $ds \rho g$ chiamo $\rho g = F$ quindi per ogni punto si ha.
Sappiamo che $T(0) = f_A , T(l) = f_b$
\begin{equation}
    \begin{aligned}
        Fds + T(s+ds) - T(s) &= 0 \\
        Fds + dT &= 0 \\
        -\frac{dT}{ds} &= F
    \end{aligned}
\end{equation}
Ora, $T(s)$ è di direzione tangente alla curva, quindi rispetto alle coordinate cartesiane, è in direzione $(\frac{dx}{ds}, \frac{dy}{ds})$
\begin{equation}
    \begin{cases}
        \displaystyle -\frac{d}{ds}\left(T(s)\frac{dx}{ds}\right) = 0 \\ \\
        \displaystyle \frac{d}{ds}\left(T(s)\frac{dy}{ds}\right) = 0
    \end{cases}
    \begin{cases}
        \displaystyle T(s)\frac{dx}{ds} = \phi \\ \\
        \displaystyle T(s)\frac{dy}{ds} = F
    \end{cases}
\end{equation}
Sostituendo
\begin{equation}
    \displaystyle \frac{d}{ds} \left(\frac{dy}{dx}\right) = \frac{F}{\phi}
\end{equation}
Un segmento $ds = dx \sqrt{1 + y'^2}$ e $\alpha = \rho g / \phi$
\begin{equation}
    \displaystyle \frac{dy'}{\sqrt{1 + y'^2}} = \alpha dx
\end{equation}
Risolvendo per per $y'$ e integrando risulta
\begin{equation}
    y(x) = \frac{1}{\alpha} \cosh(\alpha x)
\end{equation}

\subsection{Balancing the stick 2.16}
Un bastone semi-infinito ha la proprietà che in qualsiasi punto venga tagliato il resto del bastone 
riesce a stare in equilibrio se posto un cuneo a una distanza $l$ da dove viene tagliato.
Definiamo $\rho(x)$ la densità lineare della sbarra nel punto $x$ e $M(x)$ la massa della stessa fino ad x. 
Avremo quindi $M(0) = 0$ $M(+\infty) = M$ $\rho(x) = \frac{dM(x)}{dx}$
A questo punto calcoliamo il centro di massa, che deve coincidere con $l$.
\begin{equation}
    \begin{aligned}
        l &= \frac{ \int_{0}^{+\infty} \rho(x) x \,dx }{ \int_{0}^{+\infty} \rho(x) \,dx }\\
         &= \frac{ \int_{0}^{+\infty} \rho(x) x \,dx }{M}
    \end{aligned}
\end{equation}

Ciò deve diventare $l+d$, quando taglio di $d$ la sbarra
\begin{equation}
    \begin{aligned}
        l + d &= \frac{ \int_{0}^{+\infty} \rho(x) x \,dx - \int_{0}^{d} \rho(x) x \,dx }{  \int_{0}^{+\infty} \rho(x) \,dx - \int_{0}^{d} \rho(x) \,dx }\\
         &= \frac{ Ml - \int_{0}^{d} \rho(x) x \,dx }{M -  \int_{0}^{d} \rho(x) \,dx }
    \end{aligned}
\end{equation}
Ora possiamo moltiplicare e applicare gli integrali per parti sull'integrando $\rho(x) x$, ricordando che $\rho(x)$ è la derivata della funzione $M(x)$
\begin{equation}
    \begin{aligned}
        (M -  M(d) )(l + d) &=  Ml - (|M(x)x|_0^{d} - \int_{0}^{d} M(x) \,dx) \\
        Ml + Md - M(d)l - M(d)d &=  Ml - M(d)d + \int_{0}^{d} M(x) \,dx \\
        Md - M(d)l &= \int_{0}^{d} M(x) \,dx \\
    \end{aligned}
\end{equation}
L'ultima è un equazione differenziale, con soluzione
\begin{equation}
    M(d) = M(1 - e^{-d/l})
\end{equation}
Derivando, e cambiando la variabile con $x$ si trova l'espressione della densità
\begin{equation}
    \rho(x) = \frac{M}{l} e^{-x/l}
\end{equation}
\end{document}